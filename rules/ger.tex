\documentclass[10pt,a4paper]{article}
\usepackage{amssymb,amsmath}
\usepackage[utf8]{inputenc}
\usepackage{hyperref}
\usepackage[margin=0.9in]{geometry}
\addtolength{\topmargin}{-.2in}
\providecommand{\tightlist}{%
  \setlength{\itemsep}{0pt}\setlength{\parskip}{0pt}}
\setcounter{secnumdepth}{0}
\date{}

\begin{document}

\section{Nyan Nyan}\label{nyan-nyan}

\subsection{Algemein}\label{algemein}

Nyan Nyan wird mit einer beliebigen Anzahl von vollen Blättern (52
Karten + 2-4 Joker) gespielt. Am Anfang zieht jeder Spieler 7 Karten und
die oberste Karte des \emph{Deck} wird offen auf den \emph{Ablagestapel}
gelegt. Das Ziel des Spiels ist es all seine Karten loszuwerden, ähnlich
wie in \href{https://de.wikipedia.org/wiki/Uno_(Kartenspiel)}{Uno} und
\href{https://de.wikipedia.org/wiki/Mau-Mau_(Kartenspiel)}{Mau Mau}.
Wenn man am Zug ist, kann man eine Karte, die den selben Wert oder die
selbe \emph{Farbe} (\emph{Farbe} bezieht sich auf die vier Symbole
:spades::clubs::heart::diamonds:) wie die oberste Karte des
\emph{Ablegestapels} hat, von seiner \emph{Hand} auf den
\emph{Ablegestapel} legen. Wenn man nicht Spielt muss man eine Karte
ziehen (Jeder Spieler darf pro Zug nur einmal ziehen. Nachdem man
gezogen hat, hat man nochmal die Chance eine Karte zu spielen. Dannach
ist der nächste Spieler an der Reihe.

\subsection{Gewinnen/Verlieren}\label{gewinnenverlieren}

Wenn man nur noch eine Karte hat muss man \textbf{\emph{Nyan}} sagen.
Wenn man dies nicht tut, muss man 2 Karten ziehen. Wenn man seine letzte
Karte spielt muss man \textbf{\emph{Nyan Nyan}} sagen und beendet damit
(der jenige der zu erst \textbf{\emph{Nyan Nyan}} sagt, beendet zu
erst). Wenn man dies nicht tut muss man 7 Karten ziehen

\subsection{Exodia}\label{exodia}

Eine alternative Methode um zu gewinnen ist mit \textbf{\emph{Exodia}},
was man aus Poker auch als \textbf{\emph{Royal Flush}}
(\textbf{A},\textbf{K},\textbf{Q},\textbf{J},\textbf{10} der selben
Farbe) kennt. Fals man \textbf{\emph{Exodia}} auf der Hand hat, muss man
es einfach Vorzeigen und man beendet.

\subsubsection{+20 Karten ziehenn}\label{karten-ziehenn}

Wenn man 20 oder mehr Karten auf einmal ziehen muss, scheidet man aus
dem Spiel aus.

\subsubsection{+32 Karten Hand}\label{karten-hand}

Wenn man 32 oder mehr Karten auf der Hand hat, scheidet man aus dem
Spiel aus.

\subsubsection{Leeres Deck}\label{leeres-deck}

Wenn das \emph{Deck} leer ist, werden alle Karten, bis auf die oberste,
des \emph{Ablegestapels} gemischt und als neues \emph{Deck} verwendet.
Ist dies nicht möglich scheidet der Spieler mit den meisten Karten aus
und seine Hand wird gemischt und als neues \emph{Deck} verwendet.

Der Spieler der zuerst beendet wird erster, der nächste der beendet wird
2. usw. Wer zuerst ausscheidet ist letzter, der nächste 2. letzter usw.

\subsection{Karten Effekte}\label{karten-effekte}

Alle Karten Effekte werden aktiviert, wenn die Karte gespielt wird.
(Ausser der Effekt sagt anders)\\
Diese Liste ist aufsteigend nach Wert sortiert

\subsubsection{2-doppeltes Aussetzen}\label{doppeltes-aussetzen}

\begin{itemize}
\item
  Die nächsten 2 n Spieler setzen aus
\item
  n wir auf 1 gesetzt
\end{itemize}

\subsubsection{3-3er Runde}\label{er-runde}

\begin{itemize}
\item
  Es darf nur \textbf{3} gespielt werden
\item
  Der Effekt dieser Karte wird durch die nä \textbf{3} aufgehoben
\item
  Die nächste \textbf{3} hat keinen Effekt
\item
  Nach n Runden ist der Effekt automatisch aufgehoben
\item
  n wird auf 1 gesetzt
\end{itemize}

\subsubsection{4-4 Ziehen}\label{ziehen}

\begin{itemize}
\item
  Der nächste Spieler muss \textbf{4} n Karten ziehen
\item
  n wird auf 1 gesetzt
\end{itemize}

\subsubsection{5-Fusion}\label{fusion}

\begin{itemize}
\item
  Man muss n belibige Karten auf die \textbf{5} legen
\item
  Diese Karten haben keinen Effekt
\item
  Wenn man 2 \textbf{5} auf einmal spielt darf man 2 n Karten drauflegen
  (nicht bis zu)
\item
  n wird auf 1 gesetzt
\end{itemize}

\subsubsection{6-Stechen}\label{stechen}

\begin{itemize}
\item
  Jeder Spieler legt n Karten verdeckt auf den Tisch
\item
  Wenn jeder seine Karten gelegt hat, werden aufgedeckt
\item
  Der Spieler dessen höchste Karte den kleinsten Wert hat, verliert das
  Stechen
\item
  Wenn 2 oder mehr Spieler den kleinsten Wert haben, vergleichen diese
  die nächst kleinere bis ein Verlierer ermittelt werden konnte
\item
  Wenn keinen Verlierer ermittelt werden kann, spielen alle Spieler eine
  neue Karte verdeckt (die alten Karten zählen immer noch)
\item
  Dies wird solange wiederholt bis ein verlierer fest steht
\item
  Der Verlierer des Stechen muss alle Karten aufnehmen
\item
  Falls ein Spieler nicht genug Karten für das Stechen hat muss er so
  lange ziehen bis er ausreichend karten hat
\item
  n wird auf 1 gesetzt
\end{itemize}

\subsubsection{7-2 Ziehen}\label{ziehen-1}

\begin{itemize}
\item
  Der nächste Spieler muss 2 n Karten ziehen
\item
  Wenn der vorherige Spieler eine \textbf{7} gespielt hat, kann man
  anstelle zu ziehen eine \textbf{7} spielen. Dadurch muss der nächste
  Spieler 2 Karten mehr ziehen als man hätte sollen
\item
  n wird auf 1 gesetzt
\end{itemize}

\subsubsection{8-Aussetzen}\label{aussetzen}

\begin{itemize}
\item
  Die nächsten n Spieler setzen aus
\item
  n wird auf 1 gesetzt
\end{itemize}

\subsubsection{9-Richtungswechsel}\label{richtungswechsel}

\begin{itemize}
\tightlist
\item
  Die Spielrichtung wird geändert
\end{itemize}

\subsubsection{10-extra Runde}\label{extra-runde}

\begin{itemize}
\item
  Man erhält n extra Züge
\item
  Extra Züge müssen Ausgeführt werden, ach wenn man keine Karten mehr
  hat
\item
  n wird auf 1 gesetzt
\end{itemize}

\subsubsection{J-Farbwunsch}\label{j-farbwunsch}

\begin{itemize}
\tightlist
\item
  Diese Karte kann unabhängig der aktuellen Farbe gespielt werden
\item
  Man darf auswählen welche die aktive Farbe sein soll
\end{itemize}

\subsubsection{Q-Hofdame}\label{q-hofdame}

\begin{itemize}
\tightlist
\item
  Diese Karte kann unabhängig der aktuellen Farbe gespielt werden
\end{itemize}

\subsubsection{K-königlicher Befehl}\label{k-kuxf6niglicher-befehl}

\begin{itemize}
\item
  Man muss n \textbf{Befehle} verteilen
\item
  Wenn ein Spieler von einem \textbf{Befehl} getroffen ist, beendet sein
  Zug automatisch, sobald er dran kommt (dadurch wird der
  \textbf{Befehl} aufgehoben)
\item
  \textbf{4} und \textbf{7} werden durch den \textbf{Befehl} auf den
  nächsten Spieler weitergeleitet
\item
  n wird auf 1 gesetzt
\end{itemize}

\subsubsection{A-Amplifier}\label{a-amplifier}

\begin{itemize}
\item
  n wird verdoppelt
\end{itemize}

\subsubsection{$\bigstar$-Joker}\label{joker}

\begin{itemize}
\item
  Diese Karte kann unabhängig delatex starr Farbe gespielt werden
\item
  Wenn diese Karte die oberste Karte ist, zählt die Karte unter dieser
  als oberste
\item
  Diese Karte kann gespielt werden wenn man von einem Effect getroffen
  wird (\textbf{4},\textbf{6}, \textbf{7}, \textbf{K})
\item
  Zieh Effekte werden auf den nächsten Spieler übertragen
\item
  Wenn der \textbf{$\bigstar$} als Reaktion auf \textbf{K} gespielt wird, wird
  ein \textbf{Befehl} aufgehoben
\item
  Als Reaktion auf \textbf{6} enthält man sich aus dem Stechen
\item
  Diese Karte darf auch gespielt werden, wenn eine \textbf{3er Runde}
  aktiv ist
\item
  Sollte $\bigstar$ als 1. Karte liegen darf eine belibige Karte gepielt werden
\item
  n wird auf 1 gesetzt
\end{itemize}

\end{document}
